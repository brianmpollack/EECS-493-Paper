% http://www.mrunix.de/forums/showthread.php?t=59711

% 01-07-2008, 16:06
% grundoptimismus
% Registrierter Benutzer
% Registriert seit: 01.07.2008
% Beitr�ge: 6
% 
% Fehlermeldung bei \subfigure
% 
% Hallo,
% 
% ich schreib mittlerweile meine dritte gro�e Arbeit mit LaTex und hab mich auch schon ganz gut reingefuchst (Dank euerer Hilfe ), doch jetzt steh ich vor nem Problem:
% Ich will zwei Bilder mit \subfigure einbinden und bekomme folgende Fehlermeldung:
% Undefined control sequence.
% \@makesubfigurecaption ...else \captionwidthfalse \fi \setlength \captionmar...
% 
% Der Fehler wird nur angezeigt, d.h. ich kann das PDF erstellen und auch dort is alles so abgebildet, wie's sein soll. Aber der Fehler macht mich wuschig, vor allem, wenn ich net wei�, woran es liegt.
% Ich hab es auch schon mit \subfig und minipages probiert, aber das hat gleichmal gar nicht funktioniert...
% 
% Hier das Mini-Beispiel:
% \documentclass[pdftex,a4paper,pt12,bibtotoc,liststotoc]{scrreprt}
% \usepackage[ngerman]{babel}
% \usepackage[latin1]{inputenc}
% \usepackage[T1]{fontenc}
% \usepackage{amsmath}
% \usepackage{array}
% \usepackage{amssymb}
% \usepackage[pdftex]{graphicx}
% \usepackage{geometry}
% \geometry{a4paper,left=40mm,right=20mm,top=25mm,bottom=20mm}
% \usepackage{txfonts} %schrift times
% \usepackage{longtable} %f�r lange tabellen und zeilenbr�che in tab
% \usepackage{parskip} %damit abs�tze nicht einger�ckt sind
% \usepackage{epsfig} %um bilder einzuf�gen
% \usepackage{subfigure} %mehrere bilder als eine abbildung
% \usepackage{chemsym} %damit chem. Symbole nicht in $$ m�ssen
% \usepackage{wrapfig} %f�r textumflossene bilder am seitenrand
% \usepackage{lscape} %f�r seiten im querformat
% \usepackage[hang]{caption2} %f�r einger�ckte bildunterschriften
% 
% \begin{document}
% 
% \begin{figure}[h]
% \centering
% \subfigure[Femur- und Tibiakomponente \label{sfig:ymcktotal}]
% {\includegraphics[width=5cm]{ymcktotal.JPG}}
% \hspace{1cm}
% \subfigure[Patellakomponente \label{sfig:ymckpatella}]
% {\includegraphics[width=4cm]{ymckpatella.JPG}}
% \caption{YMCK}
% \label{fig:ymck}
% \end{figure}
% \end{document}
% 
% Ich hoffe, mit kann geholfen werden...
% Gr��e,
% Nadine

\documentclass[a4paper,12pt,bibtotoc,liststotoc]{scrreprt}
\usepackage[ngerman]{babel}
%\usepackage[latin1]{inputenc}
%\usepackage[T1]{fontenc}
%\usepackage{amsmath}
%\usepackage{array}
%\usepackage{amssymb}
\usepackage[demo]{graphicx}
%\usepackage{geometry}
%\geometry{a4paper,left=40mm,right=20mm,top=25mm,bottom=20mm}
%\usepackage{txfonts} %schrift times
%\usepackage{longtable} %f�r lange tabellen und zeilenbr�che in tab
%\usepackage{parskip} %damit abs�tze nicht einger�ckt sind
%\usepackage{epsfig} %um bilder einzuf�gen
\usepackage{subfigure} %mehrere bilder als eine abbildung
%\usepackage{chemsym} %damit chem. Symbole nicht in $$ m�ssen
%\usepackage{wrapfig} %f�r textumflossene bilder am seitenrand
%\usepackage{lscape} %f�r seiten im querformat
\usepackage[hang]{caption2} %f�r einger�ckte bildunterschriften

\begin{document}

\begin{figure}[h]
\centering
\subfigure[Femur- und Tibiakomponente \label{sfig:ymcktotal}]
{\includegraphics[width=5cm]{ymcktotal.JPG}}
\hspace{1cm}
\subfigure[Patellakomponente \label{sfig:ymckpatella}]
{\includegraphics[width=4cm]{ymckpatella.JPG}}
\caption{YMCK}
\label{fig:ymck}
\end{figure}
\end{document}
