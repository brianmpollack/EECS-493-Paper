% http://www.mrunix.de/forums/showthread.php?t=64250

\documentclass[a4paper, DIV11, BCOR5mm,titlepage,headsepline]{scrartcl}
\usepackage[demo]{graphicx}
\usepackage[closeFloats]{fltpage}
\usepackage{caption}

\begin{document}
\begin{FPfigure}
\ifvmode\typeout{OK!}\else\typeout{Irgs!}\fi
\begin{addmargin}{-2.3cm}
\includegraphics[width=20cm]{Bilder/flossegelE}
\par\vspace{0.2cm}
\includegraphics[width=20cm]{Bilder/hautgelE}
\par\vspace{0.2cm}
\includegraphics[width=20cm]{Bilder/kiemeE}
\par\vspace{0.2cm}
\includegraphics[width=20cm]{Bilder/magenE}
\end{addmargin}
\caption{Zweidimensionale Polyacrylamid-Gelelektrophorese von Cytosklelett-Proteinen. In der ersten Dimension erfolgte eine isoelektrische Fokussierung (IEF) und in der zweiten Dimension wurde eine SDS-PAGE durchgeführt. B: Rinderserum-Albumin; A: Kaninchen $\alpha-Aktin$. a-a‘‘, Cytoskelett-Proteine aus der Brustflosse: (a) Mit Coomassie-Blue gefärbtes Gel, (a') K8; (a'') K18; b-b'' Cytoskelett-Proteine aus der Haut: (b) Coomassie-Gel, (b') K8 , (b'') K18 , c-c'' Cytoskelett-Proteine der Kieme, (c) Coomassie-Gel, (c') K8 , (c'') K18 ; d-d'', Cytoskelett-Proteine aus dem Magen, (d) Coomassie-Gel. (d') K8 , (d'')K18. Alle Bilder wurden durch Zuschneiden und Verzerren aneinander angepasst.}
\label{fig:flossegelE}
\end{FPfigure}
\end{document}
