% http://www.golatex.de/cite-in-subfig-t2953.html

\iffalse
Ich habe jetzt eine Grafik ins Subcaption-Paket umgesetzt. Es funktioniert auch soweit so, wie ich mir das vorgestellt hatte. 
Allerdings musste ich das floatrow-Package auskommentieren, da es sich mit dem Subcaption-Package nicht vertragen hat (Fehlermeldung: Caption(s) lost.). Das brauche ich dann auch nicht unbedingt, aber ich hatte vorher mit Hilfe des Befehls Code:

\floatsetup[figure]{style=BOXED} 
 


einen Rahmen um jede Figure. Den h�tte ich gerne wieder. 
Mit \fbox bekomme ich aber nur einen Rahmen, der die Caption ausschlie�t. Was ist da die einfachste M�glichkeit bzw. was mache ich falsch? Am sch�nsten w�re es nat�rlich, wenn man global einen Rahmen um alle figures definieren k�nnte, damit ich das nicht jedes Mal da stehen habe.
\fi

\documentclass[11pt,listof=totoc,bibliography=totoc,version=first]{scrbook}

\usepackage[demo]{graphicx} %Grafikpaket

%-----------------------------
\usepackage{floatrow} %Definitionen f�r Gleitobjekte (z.B. Rahmen)
\floatsetup[figure]{style=BOXED} %Rahmen um figure Gleitobjekte und Captions

\usepackage{caption} %Bildunterschriften
\usepackage{subcaption}
%--------------------------------------------------------------------------------------------------------------------------

\begin{document}

\chapter{Kapitel}
\label{sec:Kapitel}

\section{Abschnitt}
\label{sec:Abschnitt}

\begin{figure}[htbp]
	\centering
	\begin{subfigure}[t]{.4\linewidth}
		\centering
		\includegraphics[height=4cm]{a.jpg}\newline
		%\cite{bild:EPC-Label}
		\caption{Komponente A.}
		\label{fig:KomponenteA}
	\end{subfigure}%
	\hspace{1cm}
	\begin{subfigure}[t]{.4\linewidth}
		\centering
		\includegraphics[height=4cm]{b.jpg}\newline
		%\cite{bild:EPC-Durchgangsleser}
		\caption{Komponente B.}
		\label{fig:KomponenteB}		
	\end{subfigure}
	\caption{Komponenten}
\end{figure}

\begin{figure}
	\begin{subfigure}[b]{.5\linewidth}
		\centering\large A
		\newline\caption{A subfigure}\label{fig:1a}
	\end{subfigure}%
	\begin{subfigure}[b]{.5\linewidth}
		\centering\large B
		\newline\caption{Another subfigure}\label{fig:1b}
	\end{subfigure}
	\caption{A figure}\label{fig:1}
\end{figure}

\end{document}