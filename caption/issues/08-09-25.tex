% http://www.mrunix.de/forums/showthread.php?t=61141
% Falsche Nummerierung bei "within=section" in Klassen mit \chapter

\iffalse

mrunix.de (http://www.mrunix.de/forums/index.php) 
-   LaTeX-Forum (http://www.mrunix.de/forums/forumdisplay.php?f=38) 
-   -   andere Grafiken-Nummerierung einstellen (http://www.mrunix.de/forums/showthread.php?t=61141) 
TobiD	25-09-2008 00:11

andere Grafiken-Nummerierung einstellen
 
Hallo!
Ich habe ein kleines Problem, welches ich durch die Forensuche und das Grafiken-FAQ-Pdf nicht lösen konnte.

Ich habe ein Dokument vom Typ scrreprt. Die eingefügten Grafiken werden einfach folgendermaßen nummeriert:
1.1
1.2
1.3
...
Ich würde es aber lieber folgendermaßen haben:

1 Kapitel1

1.1 Abschnitt1
-Grafik1-
Abbildung 1.1.1: ...
...
-Grafik2-
Abbildung 1.1.2: ...
1.2 Abschnitt2
-Grafik3-
Abbildung 1.2.1: ...

2 Kapitel2

2.1 Abschnitt1
-Grafik4-
Abbildung 2.1.1: ...
...

So dass auch im \listoffigures diese Nummerierung angezeigt wird.
Ich hoffe, dass ihr versteht, was ich meine. Ich finde es nämlich sinnlos, dass im Kapitel 3 eine Grafik die Bezeichnung "1.9" hat.
Kann mir da jemand weiterhelfen?

Stefan_K	25-09-2008 00:44

chngctr
 
Hallo Tobi,

willkommen im Forum!
Schau einmal hier in der UK TeX FAQ: Running equation, figure and table numbering.

Viele Grüße,

Stefan

localghost	25-09-2008 08:47

Alternative
 
Die aktuelle Version von caption bietet eine weitere Möglichkeit zur Gestaltung der Nummerierung.
Code:
\usepackage[font=small,labelfont=bf]{caption}
\captionsetup{%
  figurewithin=section,
  tablewithin=section
}
Ich weiß jetzt gerade nicht, ob KOMA Script das nicht auch bietet. Aber darüber sollte dessen Anleitung ja Auskunft geben.


MfG
Thorsten¹

TobiD	25-09-2008 21:27

Danke für die Antwort! Mit caption haut es hin - allerdings hat es erst mal eine vierstellige statt einer dreistelligen Nummer angezeigt; Ich habs wie folgt umdefiniert:
Code:
\captionsetup{figurewithin=section}
  \renewcommand\thefigure{\arabic{chapter}.\arabic{section}.\arabic{figure}}
  \captionsetup{tablewithin=section}
  \renewcommand\thetable{\arabic{chapter}.\arabic{section}.\arabic{table}}
So hauts hin. Problem gelöst :cool:

localghost	25-09-2008 21:41

Zeig her
 
Zitat:Zitat von TobiD (Beitrag 276831) 
Danke für die Antwort! Mit caption haut es hin - allerdings hat es erst mal eine vierstellige statt einer dreistelligen Nummer angezeigt […]

Zeige mir das doch bitte mal an einem Minimalbeispiel.

sommerfee	25-09-2008 21:58

Zitat:Zitat von localghost (Beitrag 276834) 
Zeige mir das doch bitte mal an einem Minimalbeispiel.

Mir auch! Denn wenn das ein Fehler im caption-Paket ist (und danach sieht es ja aus), würde ich gerne die Chance bekommen, den Fehler zu beseitigen.

Liebe Grüße,
Axel

TobiD	25-09-2008 23:03

Ok, ich werds versuchen...

Code:
\documentclass[a4paper,11pt]{scrreprt}
\usepackage[ngerman]{babel}

\usepackage{graphicx}
\usepackage{caption}
 \captionsetup{%
  figurename=Graphik,
  listfigurename=Graphikverzeichnis,
  labelfont={footnotesize,bf},
  tablewithin=section,
  figurewithin=section
  }
  
\begin{document}

\tableofcontents
\listoffigures

\chapter{Kapitel1}


\section{Bereich1}

\subsection{Unterbereich1}
\begin{figure}
\includegraphics{Bild1.eps}
\caption{Bild1}
\end{figure}
\subsection{Unterbereich2}
\begin{figure}
\includegraphics{Bild2.eps}
\caption{Bild2}
\end{figure}

\section{Bereich2}

\subsection{Unterbereich1}
\begin{figure}
\includegraphics{Bild3.eps}
\caption{Bild3}
\end{figure}

\end{document}

Man sieht außerdem, dass im Grafikverzeichnis ein Anzeigefehler der Nummerierung vorliegt.

rais	26-09-2008 09:47

Moin moin,
Zitat:Zitat von sommerfee (Beitrag 276836) 
Mir auch! Denn wenn das ein Fehler im caption-Paket ist (und danach sieht es ja aus), würde ich gerne die Chance bekommen, den Fehler zu beseitigen.

sieht wirklich so aus:
Code:
\listfiles
\documentclass{scrreprt}
\usepackage[figurewithin=section]{caption}
\begin{document}
%\show\thefigure
\addtocounter{chapter}{2}
\chapter{foo}
\section{bar}
\section{baz}
\begin{figure}
\caption{Test}
\end{figure}
\end{document}
mit
Code:
*File List*
scrreprt.cls    2008/01/30 v2.98b KOMA-Script document class (report)
scrkbase.sty    2008/01/30 v2.98b KOMA-Script package (basics and keyval use)
  keyval.sty    1999/03/16 v1.13 key=value parser (DPC)
scrlfile.sty    2007/12/18 v2.98 KOMA-Script package (loading files)
  size11.clo    2005/09/16 v1.4f Standard LaTeX file (size option)
typearea.sty    2008/01/30 v2.98b KOMA-Script package (type area)
 caption.sty    2008/08/24 v3.1j Customizing captions (AR)
caption3.sty    2008/08/24 v3.1j caption3 kernel (AR)
 ***********
gibt `Figure 3.3.2.1' -- hier sieht man auch, daß der Kapitelzähler doppelt ausgegeben wird. Mit aktiviertem \show\thefigure:
Code:
> \thefigure=macro:
->\ifnum \c@chapter >\z@ \thechapter .\fi \ifnum \c@section >\z@ \thesection .\
fi \arabic {figure}.
l.5 \show\thefigure
eben: \thechapter ist im \thesection schon mit drin.;)

MfG

sommerfee	27-09-2008 09:51

Ja, da habe ich wohl Mist gebaut. Das hat definitiv mal funktioniert, da muß bei den letzten zwischen-Tür-und-Angel-Änderungen am caption-Paket was schief gelaufen sein. Ich werde versuchen, nächstes Wochenende eine neue Version zu basteln, und im Winter werde ich dann hoffentlich etwas mehr Zeit für das caption-Paket finden, um die Anleitung nachzupflegen etc.

Gruß,
Axel


Alle Zeitangaben in WEZ +1. Es ist jetzt 12:32 Uhr.	


Powered by vBulletin® Version 3.8.6 (Deutsch)
Copyright ©2000 - 2010, Jelsoft Enterprises Ltd.
\fi


\documentclass[a4paper,11pt]{scrreprt}
\usepackage[ngerman]{babel}

\usepackage[demo]{graphicx}
\usepackage{caption}
 \captionsetup{%
   figurename=Graphik,
   listfigurename=Graphikverzeichnis,
   labelfont={footnotesize,bf},
   tablewithin=section,
   figurewithin=section
   }
   
\begin{document}

\tableofcontents
\listoffigures

\chapter{Kapitel1}


\section{Bereich1}

\subsection{Unterbereich1}
\begin{figure}
\includegraphics{Bild1.eps}
\caption{Bild1}
\end{figure}
\subsection{Unterbereich2}
\begin{figure}
\includegraphics{Bild2.eps}
\caption{Bild2}
\end{figure}

\section{Bereich2}

\subsection{Unterbereich1}
\begin{figure}
\includegraphics{Bild3.eps}
\caption{Bild3}
\end{figure}

\end{document}
