\iffalse
From: Jens Saak <jens.saak@mathematik.tu-chemnitz.de>
Organization: TU Chemnitz MiIT 
To: caption@sommerfee.de
Subject: Subcation/Subfigure und Kapitel
Date: Wed, 1 Oct 2008 12:05:34 +0200
User-Agent: KMail/1.9.6 (enterprise 20070904.708012)
Message-Id: <200810011205.38818.jens.saak@mathematik.tu-chemnitz.de>

Hallo,
anbei ein kleines Minimalbeispiel, dass meines Erachtens einen Fehler des=20
subcation Paketes bei der Verwendung von kapitelbasierten Dokumentklassen=20
aufzeigt.

Offenbar wird der subcaption counter nicht zur=FCckgesetzt, wenn es nur ein=
e=20
=46igure mit subfigures/-captions in einem Kapitel gibt.  Kommentiert man=20
dagegen die Zeile \chapter{Test2} aus , so dass das erste Kapitel die beide=
n=20
ersten Abbildungen enth=E4lt, dann funktioniert das ganze (f=FCr den n=E4ch=
sten=20
Kapitelwechsel) korrekt.

Die Testumgebung:
TexLive2008=20

Das Testfile zeigt aber bei einer =E4lteren TexLive2007-Installation mit de=
n=20
gleichen (neuesten) Versionen von caption und subcaption dasselbe Verhalten.

Ansonsten finde ich aber das Paket absolut hervorragend. subfigure und subf=
ig=20
hatten mir zuvor derart viele graue Haare beim zusammenspiel mit hyperref=20
beschert, dass es ein reiner Segen ist wie problemlos hier alles=20
funktioniert, bis eben auf diese Kleinigkeit.

Danke f=FCr dieses sch=F6ne Paket,
Jens Saak

=2D-=20
=2D----------------------------------------------------------
Jens Saak
=46akult=E4t f=FCr Mathematik
TU Chemnitz
D-09107 Chemnitz
Germany

Tel.   : (+49)(0)371 531 32142
=46ax.   : (+49)(0)371 531 22509
E-mail : jens.saak@mathematik.tu-chemnitz.de
URL    : http://www-user.tu-chemnitz.de/~saak
\fi

\listfiles
\documentclass{book}

%\usepackage[pdftex]{graphicx}
\usepackage[list=true]{caption,subcaption}
%\usepackage[pdftex]{hyperref}

\begin{document}
\listoffigures

\chapter{Test}
\begin{figure}
  \begin{subfigure}[b]{.5\linewidth}
    \centering\large A
    \caption{A subfigure}\label{fig:1a}
  \end{subfigure}%
  \begin{subfigure}[b]{.5\linewidth}
    \centering\large B
    \caption{Another subfigure}\label{fig:1b}
  \end{subfigure}
  \caption{A figure}\label{fig:1}
\end{figure}


%\chapter{Test2}
\begin{figure}
  \begin{subfigure}[b]{.5\linewidth}
    \centering\large A
    \caption{A  second subfigure}\label{fig:2a}
  \end{subfigure}%
  \begin{subfigure}[b]{.5\linewidth}
    \centering\large B
    \caption{Another second subfigure}\label{fig:2b}
  \end{subfigure}
  \caption{A second figure}\label{fig:2}
\end{figure}

\chapter{Test3}
\begin{figure}
  \begin{subfigure}[b]{.5\linewidth}
    \centering\large A
    \caption{A third subfigure}\label{fig:3a}
  \end{subfigure}%
  \begin{subfigure}[b]{.5\linewidth}
    \centering\large B
    \caption{Another third subfigure}\label{fig:3b}
  \end{subfigure}
  \caption{A third figure}\label{fig:3}
\end{figure}

\chapter{Test4}
\begin{figure}
  \begin{subfigure}[b]{.5\linewidth}
    \centering\large A
    \caption{A fourth subfigure}\label{fig:4a}
  \end{subfigure}%
  \begin{subfigure}[b]{.5\linewidth}
    \centering\large B
    \caption{Another fourth subfigure}\label{fig:4b}
  \end{subfigure}
  \caption{A fourth figure}\label{fig:4}
\end{figure}
\end{document}

%%% Local Variables:
%%% mode: TEX-PDF
%%% TeX-master: t
%%% End:
